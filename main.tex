\documentclass{article}
\usepackage[utf8]{inputenc}
\usepackage[T1]{fontenc}
\usepackage{url}
\usepackage{amsmath, amssymb}
\usepackage{float}
\usepackage{tikz}
\usepackage{subcaption}
\usetikzlibrary{positioning, arrows.meta}

\title{Vehicle Routing Problem - VRP}
\author{Emma Szigeti, Flaviu Sălnicean}
\date{Octombrie 2025}

\begin{document}

\maketitle

\section{Introduction}

\textbf{...........................................................................................................}

\section{Difference between the Vehicle Routing Problem and the Travelling Salesman Problem}

The Travelling Salesman Problem can be considered, at least from my point of view, as a simplified version of the Vehicle Routing Problem.

\textbf{The Travelling Salesman Problem (TSP)} consists in finding the shortest possible route that passes through a given set of points (theoretically, cities), each visited exactly once, with known distances between them, and the route must end at the starting point.

\textbf{The Vehicle Routing Problem (VRP)}, in its basic form, consists in finding a set of routes with minimal total cost for a fleet of vehicles (theoretically, delivery trucks) starting from the same depot, such that each customer is visited exactly once. Various additional constraints can be added later.

\textbf{So, what is the difference?} In the Travelling Salesman Problem, there is only one person who must complete the shortest possible tour through all points, while in the Vehicle Routing Problem there are multiple vehicles, each requiring an optimal route that covers a subset of points so that, together, all points are visited.

\section{Types of VRP}

There are several variations of the Vehicle Routing Problem, including:
\begin{itemize}
    \item CVRP (Capacitated Vehicle Routing Problem)
    \item VRPP (Vehicle Routing Problem with Profits)
    \item VRPTW (Vehicle Routing Problem with Time Windows)
    \item VRPWT (Vehicle Routing Problem with Transfers)
    \item VRPMT (Vehicle Routing Problem with Multiple Trips)
    \item VRPPD (Vehicle Routing Problem with Pickup and Delivery)
    \item VRPB (Vehicle Routing Problem with Backhauls)
    \item PVRP (Periodic Vehicle Routing Problem)
    \item SVRP (Stochastic Vehicle Routing Problem)
    \item SDVRP (Split Delivery Vehicle Routing Problem)
    \item MDVRP (Multiple Depot Vehicle Routing Problem)
    \item IRP (Inventory Routing Problem)
    \item OVRP (Open Vehicle Routing Problem)
    \item OP (Orienteering Problem)
    \item TOP (Team Orienteering Problem)
    \item CTOP (Capacitated Team Orienteering Problem)
    \item TOPTW (Capacitated Team Orienteering Problem with Time Windows)
    \item EVRP (Electric Vehicle Routing Problem)
\end{itemize}

\section{The Meaning of Each Variant}

\textbf{CVRP} is the simplest and most common form of VRP. Each vehicle has a limited carrying capacity, and the goal is to find the set of routes with minimal total cost.

\textbf{VRPP} is a variant of VRP in which each customer provides a certain profit, while each trip has an associated cost. The objective is to maximize the total profit.

\textbf{VRPTW} introduces time windows for customers: each must be visited within a specific time interval. The goal remains to minimize the total cost.

\textbf{VRPWT} allows vehicles to leave or pick up goods at transfer points. The objective is to find routes with minimal total cost.

\textbf{VRPMT} allows each vehicle to perform multiple trips per day (for instance, returning to the depot to reload). The goal is again to minimize total travel cost.

\textbf{VRPPD} considers customers who both receive and send deliveries. When a vehicle visits a customer, it delivers a package and picks up another one for delivery elsewhere. The goal is to minimize the total cost.

\textbf{VRPB} distinguishes between two types of customers: those who only receive goods and those who only send them. The objective is to minimize total cost.

\textbf{PVRP} extends VRP over a planning horizon (for example, a week or a month). The goal is to decide on which days and with which vehicles deliveries will be made to each customer.

\textbf{SVRP} introduces randomness: some parameters are uncertain (stochastic), such as traffic conditions, weather, or customer demand. The objective is to minimize expected total cost while accounting for uncertainty.

\textbf{SDVRP} allows customer demands to be split among several vehicles, meaning that deliveries can be divided. The goal is to minimize total cost.

\textbf{MDVRP} involves multiple depots from which deliveries can be made to customers. The goal is to minimize total transportation cost.

\textbf{IRP} integrates inventory management. Each customer has a minimum and maximum stock level, and the objective is to plan routes that maintain proper inventory levels while minimizing total cost.

\textbf{OVRP} assumes that vehicles do not return to the depot at the end of the day. The objective remains to minimize total cost.

\textbf{OP} (Orienteering Problem) associates each customer with a profit, and a single vehicle has a limited distance or time available. The goal is to select which customers to visit to maximize total profit.

\textbf{TOP} (Team Orienteering Problem) generalizes the OP to multiple vehicles. The goal is to select customer visits that maximize total profit.

\textbf{CTOP} adds capacity constraints to the TOP, where each vehicle has limited capacity. The objective remains to maximize total profit.

\textbf{TOPTW} extends CTOP by adding time windows for customers. The goal is to select customer visits that maximize total profit while respecting time constraints.

\textbf{EVRP} models routes for electric vehicles with limited driving range and recharging requirements. The goal is to minimize total cost while considering energy consumption and charging time.

\section{Chosen Variant}

For this project, we chose to work with the CVRP (Capacitated Vehicle Routing Problem) variant, as it is the easiest to understand and to implement.

\section{Formulare matematică pentru Capacitated Vehicle Routing Problem (CVRP)}

\textbf{Date:}
\begin{itemize}
  \item $V = \{0, 1, \dots, n\}$ --- mulțimea nodurilor, unde $0$ este depozitul și $1,\dots,n$ sunt clienții.
  \item $c_{ij} \ge 0$ --- costul (sau distanța) de la $i$ la $j$.
  \item $d_i > 0$ --- cererea clientului $i$, pentru $i = 1,\dots,n$; $d_0 = 0$.
  \item $Q > 0$ --- capacitatea fiecărui vehicul.
  \item $K$ --- numărul maxim de vehicule disponibile.
\end{itemize}

\textbf{Variabile de decizie:}
\[
x_{ij} \in \{0,1\}, \quad \forall i,j \in V, \ i \neq j
\]
\[
f_{ij} \ge 0, \quad \forall i,j \in V, \ i \neq j
\]

\textit{Interpretare:} $x_{ij} = 1$ dacă arcul $(i,j)$ este utilizat într-o rută; 
$f_{ij}$ este fluxul de marfă (încărcătura) transportat pe arc-ul $i \to j$.
\bigskip

\textbf{Funcția obiectiv:}
\[
\min \sum_{i \in V} \sum_{\substack{j \in V \\ j \neq i}} c_{ij} \, x_{ij}
\]

\textbf{Constrângeri:}
\begin{align}
\sum_{\substack{j \in V \\ j \neq i}} x_{ij} &= 1, && \forall i \in \{1,\dots,n\} 
\quad &&\text{(fiecare client este vizitat o dată)} \\[4pt]
\sum_{\substack{i \in V \\ i \neq j}} x_{ij} &= 1, && \forall j \in \{1,\dots,n\} 
\quad &&\text{(fiecare client este părăsit o dată)} \\[4pt]
\sum_{\substack{j \in V \\ j \neq 0}} x_{0j} &\le K, \quad
\sum_{\substack{i \in V \\ i \neq 0}} x_{i0} \le K 
&& &&\text{(maxim $K$ vehicule)} \\[4pt]
0 \le f_{ij} &\le Q \, x_{ij}, && \forall i,j \in V, \ i \neq j 
\quad &&\text{(legătura flux--arc)} \\[4pt]
\sum_{\substack{j \in V \\ j \neq i}} f_{ji} + d_i 
&= \sum_{\substack{j \in V \\ j \neq i}} f_{ij}, && \forall i \in \{1,\dots,n\}
\quad &&\text{(conservare flux la clienți)} \\[4pt]
\sum_{\substack{j \in V \\ j \neq 0}} f_{0j} &= \sum_{i=1}^{n} d_i 
\quad &&\text{(flux total pornit din depozit)}
\end{align}

Domenii:
\[
x_{ij} \in \{0,1\}, \qquad f_{ij} \ge 0
\]

\bigskip

\section{Funcția de fitness și costul penalizat pentru CVRP}
În cadrul algoritmilor genetici, funcția de fitness are rolul de a evalua calitatea unei soluții candidate (set de rute). Deoarece nu toate soluțiile generate respectă constrângerile de capacitate sau numărul de vehicule, se utilizează un cost penalizat care integrează aceste abateri în valoarea totală a costului.

\bigskip

\textbf{Fie pentru o soluţie $S$ (set de rute) costul total nepenalizat:}
\[
C(S)\;=\;\sum_{r\in S}\;\sum_{(i,j)\in r} c_{ij},
\]
unde suma internă se face pe arcele succesive ale rutei $r$ (inclusiv arcul de la depozit 0 la primul client şi arcul de la ultimul client înapoi la 0).

\section*{Definiţii pe rută}
Pentru fiecare rută $r\in S$ definim:
\[
Q_r \;=\; \sum_{i\in r} d_i
\qquad\text{(încărcătura totală pe ruta }r\text{)},
\]
\[
\Delta_r \;=\; \max\bigl(0,\;Q_r - Q\bigr)
\qquad\text{(depăşirea capacităţii pentru ruta }r\text{)}.
\]

Dacă există o limită $K$ pe numărul de vehicule, definim şi:
\[
\Delta_K \;=\; \max\bigl(0,\; |S| - K \bigr)
\qquad\text{(numărul de vehicule în exces)}.
\]

\section*{Cost penalizat}
Atunci costul penalizat pentru soluţia $S$ este:
\[
C_{\mathrm{pen}}(S)
\;=\;
C(S) \;+\; \alpha \sum_{r\in S} \Delta_r \;+\; \beta\,\Delta_K,
\]
unde $\alpha,\beta>0$ sunt coeficienţi de penalizare (de exemplu: $\alpha$ — cost pe unitate de marfă depăşită; $\beta$ — cost fix per vehicul în exces).

\section*{Transformare în funcţie de fitness}
\[
\operatorname{fitness}(S)
\;=\;
\frac{1}{1 + C_{\mathrm{pen}}(S)}.
\]
Astfel, valorile de fitness mai mari corespund soluțiilor cu cost penalizat mai mic (adică rute mai scurte și mai fezabile).

\bigskip

\section{Representation of the Individual for CVRP}

In the Genetic Algorithm framework, each individual encodes a possible solution to the Capacitated Vehicle Routing Problem (CVRP). 
Each solution consists of a set of routes that start and end at the central depot, visiting every customer exactly once.

\subsection*{Array-based sequential representation}

In this representation, each gene corresponds to a node: either the depot or a customer. 
The value 0 represents the depot (start or end of a route), while numbers $1,2,3,\dots,n$ represent customers. 

For example:
\[
\begin{aligned}
R_1 &: 0 \rightarrow 2 \rightarrow 5 \rightarrow 7 \rightarrow 0, \\
R_2 &: 0 \rightarrow 3 \rightarrow 4 \rightarrow 8 \rightarrow 0, \\
R_3 &: 0 \rightarrow 1 \rightarrow 6 \rightarrow 9 \rightarrow 0.
\end{aligned}
\]

\subsection*{Advantages}
\begin{itemize}
  \item Clear and intuitive structure: The representation directly shows the order of visits for each vehicle.
  \item Easy to verify route validity and vehicle capacity constraints: You can quickly compute the load of each vehicle.
  \item Single-array representation: All routes are represented in a single array, simplifying storage and manipulation.
  \item Flexible for different VRP variants: Works with CVRP, VRPTW, VRPPD, etc., by adjusting constraints or penalties.
\end{itemize}

\subsection*{Disadvantages}
\begin{itemize}
  \item Sequence interpretation can be less intuitive for very large instances: Long arrays with many vehicles and customers can be harder to read manually.
  \item Fixed depot representation: If multiple depots are involved, the array format needs modification.
  \item Limited direct information about route optimization: The representation shows order, but does not encode distance minimization explicitly.
\end{itemize}

\subsection*{Representation of a CVRP solution using numerical customer IDs}

\begin{figure}[H]
\centering

\caption*{
Each number represents a customer, and 0 represents the depot (start and end of each route). 
The sequence encodes the exact visiting order of customers for each route.
}

\vspace{1cm}

\begin{subfigure}{\textwidth}
\centering
\begin{tikzpicture}[
    every node/.style={circle, draw, thick, minimum size=1cm, font=\bfseries},
    depot/.style={fill=red!80, text=white},
    groupA/.style={fill=cyan!40},
    groupB/.style={fill=orange!70},
    groupC/.style={fill=green!40}
]

\node[depot] (depot) {depot};
\node[groupA] (1) [above left=1.5cm and 1cm of depot] {1};
\node[groupA] (2) [above=1.5cm of depot] {2};
\node[groupA] (7) [above right=1.5cm and 1cm of depot] {7};
\node[groupC] (5) [right=1.5cm of depot] {5};
\node[groupC] (6) [below right=1.5cm and 1cm of depot] {6};
\node[groupB] (4) [below left=1.5cm and 1cm of depot] {4};
\node[groupB] (3) [left=1.5cm of depot] {3};

\draw (depot) -- (1);
\draw (depot) -- (2);
\draw (depot) -- (7);
\draw (depot) -- (3);
\draw (depot) -- (4);
\draw (depot) -- (5);
\draw (depot) -- (6);
\draw (1) -- (2);
\draw (2) -- (7);
\draw (5) -- (6);
\draw (3) -- (4);

\node[below=0.1cm of depot, draw=none, inner sep=0pt] (seq) {
\begin{tikzpicture}[every node/.style={rectangle, draw=black, thick,
                                       minimum width=0.8cm, minimum height=0.8cm,
                                       inner sep=0pt, outer sep=0pt, font=\bfseries},
                    groupA/.style={fill=cyan!40},
                    groupB/.style={fill=orange!70},
                    groupC/.style={fill=green!40}]
]
    \node (a0) {0};
    \node[groupC, right=0pt of a0] (a5) {5};
    \node[groupC, right=0pt of a5] (a6) {6};
    \node[right=0pt of a6] (a00) {0};
    \node[groupA, right=0pt of a00] (a1) {1};
    \node[groupA, right=0pt of a1] (a2) {2};
    \node[groupA, right=0pt of a2] (a7) {7};
    \node[right=0pt of a7] (a01) {0};
    \node[groupB, right=0pt of a01] (a3) {3};
    \node[groupB, right=0pt of a3] (a4) {4};
    \node[right=0pt of a4] (a02) {0};
\end{tikzpicture}
};

\end{tikzpicture}
\vspace{-2.5cm}
\caption*{(1) Visual grouping of customers by color and route}
\end{subfigure}

\label{fig:cvrp_chromosome_clients}
\end{figure}

\newpage
\section{Bibliografie}
\url{https://blog.locus.sh/vehicle-routing-problem/}

\url{https://en.wikipedia.org/wiki/Vehicle_routing_problem}

\url{https://tttp-au.com/wp-content/uploads/2024/06/Pages-from-TTTP-Vol-9_No-1_2024-WEB-8.pdf}

\url{https://ro.wikipedia.org/wiki/Problema_rut%C4%83rii_vehiculelor}

\url{https://ro.wikipedia.org/wiki/Problema_comis-voiajorului}

\url{https://developers.google.com/optimization/routing/cvrp}

\url{https://www.upperinc.com/glossary/route-optimization/vehicle-routing-problem-with-profits-vrpp/}

\url{https://developers.google.com/optimization/routing/vrptw}

\url{https://www.sciencedirect.com/science/article/abs/pii/S0305054825000085}

\url{https://www.sciencedirect.com/science/article/abs/pii/S0305054820301040}

\url{https://www.upperinc.com/glossary/route-optimization/vehicle-routing-problem-with-backhauls-vrpb/}

\url{https://www.upperinc.com/glossary/route-optimization/periodic-vehicle-routing-problem-pvrp/}

\url{https://www.upperinc.com/glossary/route-optimization/stochastic-vehicle-routing-problem-svrp/}

\url{https://www.sciencedirect.com/science/article/pii/S0305054814002159}

\url{https://www.sciencedirect.com/science/article/abs/pii/S0305054813001408}

\url{https://www.hexaly.com/templates/inventory-routing-problem-irp}

\url{https://www.upperinc.com/glossary/route-optimization/open-vehicle-routing-problem-ovrp/}

\url{https://www.sciencedirect.com/topics/mathematics/orienteering-problem}

\url{https://drops.dagstuhl.de/storage/01oasics/oasics-vol014-atmos2010/OASIcs.ATMOS.2010.142/OASIcs.ATMOS.2010.142.pdf}

\url{https://www.researchgate.net/publication/256022425_The_Capacitated_Team_Orienteering_Problem_A_Bi-Level_Filter-and-Fan_Method}

\url{https://www.praiseworthyprize.org/jsm/index.php?journal=irecos&page=article&op=view&path%5B%5D=18244}

\url{https://www.sciencedirect.com/science/article/pii/S2352146516000089}

\url{https://www.researchgate.net/figure/A-CVRP-solution-and-the-individual-encoding-representation-Each-color-represents-a_fig2_394965948}

\end{document}