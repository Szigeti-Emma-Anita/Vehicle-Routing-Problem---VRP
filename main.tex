\documentclass{article}
\usepackage[utf8]{inputenc}
\usepackage[T1]{fontenc}
\usepackage{url}
\usepackage{amsmath, amssymb}
\usepackage{float}
\usepackage{tikz}
\usepackage{subcaption}
\usetikzlibrary{positioning, arrows.meta}

\title{Vehicle Routing Problem - VRP}
\author{Emma Szigeti, Flaviu Sălnicean}
\date{Octombrie 2025}

\begin{document}

\maketitle

\section{Introduction}

\textbf{...........................................................................................................}

\section{Difference between the Vehicle Routing Problem and the Travelling Salesman Problem}

The Travelling Salesman Problem can be considered, at least from my point of view, as a simplified version of the Vehicle Routing Problem.

\textbf{The Travelling Salesman Problem (TSP)} consists in finding the shortest possible route that passes through a given set of points (theoretically, cities), each visited exactly once, with known distances between them, and the route must end at the starting point.

\textbf{The Vehicle Routing Problem (VRP)}, in its basic form, consists in finding a set of routes with minimal total cost for a fleet of vehicles (theoretically, delivery trucks) starting from the same depot, such that each customer is visited exactly once. Various additional constraints can be added later.

\textbf{So, what is the difference?} In the Travelling Salesman Problem, there is only one person who must complete the shortest possible tour through all points, while in the Vehicle Routing Problem there are multiple vehicles, each requiring an optimal route that covers a subset of points so that, together, all points are visited.

\section{Types of VRP}

There are several variations of the Vehicle Routing Problem, including:
\begin{itemize}
    \item CVRP (Capacitated Vehicle Routing Problem)
    \item VRPP (Vehicle Routing Problem with Profits)
    \item VRPTW (Vehicle Routing Problem with Time Windows)
    \item VRPWT (Vehicle Routing Problem with Transfers)
    \item VRPMT (Vehicle Routing Problem with Multiple Trips)
    \item VRPPD (Vehicle Routing Problem with Pickup and Delivery)
    \item VRPB (Vehicle Routing Problem with Backhauls)
    \item PVRP (Periodic Vehicle Routing Problem)
    \item SVRP (Stochastic Vehicle Routing Problem)
    \item SDVRP (Split Delivery Vehicle Routing Problem)
    \item MDVRP (Multiple Depot Vehicle Routing Problem)
    \item IRP (Inventory Routing Problem)
    \item OVRP (Open Vehicle Routing Problem)
    \item OP (Orienteering Problem)
    \item TOP (Team Orienteering Problem)
    \item CTOP (Capacitated Team Orienteering Problem)
    \item TOPTW (Capacitated Team Orienteering Problem with Time Windows)
    \item EVRP (Electric Vehicle Routing Problem)
\end{itemize}

\section{The Meaning of Each Variant}

\textbf{CVRP} is the simplest and most common form of VRP. Each vehicle has a limited carrying capacity, and the goal is to find the set of routes with minimal total cost.

\textbf{VRPP} is a variant of VRP in which each customer provides a certain profit, while each trip has an associated cost. The objective is to maximize the total profit.

\textbf{VRPTW} introduces time windows for customers: each must be visited within a specific time interval. The goal remains to minimize the total cost.

\textbf{VRPWT} allows vehicles to leave or pick up goods at transfer points. The objective is to find routes with minimal total cost.

\textbf{VRPMT} allows each vehicle to perform multiple trips per day (for instance, returning to the depot to reload). The goal is again to minimize total travel cost.

\textbf{VRPPD} considers customers who both receive and send deliveries. When a vehicle visits a customer, it delivers a package and picks up another one for delivery elsewhere. The goal is to minimize the total cost.

\textbf{VRPB} distinguishes between two types of customers: those who only receive goods and those who only send them. The objective is to minimize total cost.

\textbf{PVRP} extends VRP over a planning horizon (for example, a week or a month). The goal is to decide on which days and with which vehicles deliveries will be made to each customer.

\textbf{SVRP} introduces randomness: some parameters are uncertain (stochastic), such as traffic conditions, weather, or customer demand. The objective is to minimize expected total cost while accounting for uncertainty.

\textbf{SDVRP} allows customer demands to be split among several vehicles, meaning that deliveries can be divided. The goal is to minimize total cost.

\textbf{MDVRP} involves multiple depots from which deliveries can be made to customers. The goal is to minimize total transportation cost.

\textbf{IRP} integrates inventory management. Each customer has a minimum and maximum stock level, and the objective is to plan routes that maintain proper inventory levels while minimizing total cost.

\textbf{OVRP} assumes that vehicles do not return to the depot at the end of the day. The objective remains to minimize total cost.

\textbf{OP} (Orienteering Problem) associates each customer with a profit, and a single vehicle has a limited distance or time available. The goal is to select which customers to visit to maximize total profit.

\textbf{TOP} (Team Orienteering Problem) generalizes the OP to multiple vehicles. The goal is to select customer visits that maximize total profit.

\textbf{CTOP} adds capacity constraints to the TOP, where each vehicle has limited capacity. The objective remains to maximize total profit.

\textbf{TOPTW} extends CTOP by adding time windows for customers. The goal is to select customer visits that maximize total profit while respecting time constraints.

\textbf{EVRP} models routes for electric vehicles with limited driving range and recharging requirements. The goal is to minimize total cost while considering energy consumption and charging time.

\section{Chosen Variant}

For this project, we chose to work with the CVRP (Capacitated Vehicle Routing Problem) variant, as it is the easiest to understand and to implement.

\section{Mathematical formulation for the Capacitated Vehicle Routing Problem (CVRP)}

\textbf{Data:}
\begin{itemize}
\item $V = {0, 1, \dots, n}$ --- set of nodes, where $0$ is the depot and $1,\dots,n$ are the customers.
\item $c_{ij} \ge 0$ --- cost (or distance) from $i$ to $j$.
\item $d_i > 0$ --- demand of customer $i$, for $i = 1,\dots,n$; $d_0 = 0$.
\item $Q > 0$ --- capacity of each vehicle.
\item $K$ --- maximum number of available vehicles.
\end{itemize}

\textbf{Decision variables:}
\[
x_{ij} \in \{0,1\}, \quad \forall i,j \in V, \ i \neq j
\]
\[
f_{ij} \ge 0, \quad \forall i,j \in V, \ i \neq j
\]

\textit{Interpretation:} $x_{ij} = 1$ if the arc $(i,j)$ is used in a route;
$f_{ij}$ is the flow of goods transported on arc $i \to j$.

\bigskip

\textbf{Objective function:}
\[
\min \sum_{i \in V} \sum_{\substack{j \in V \\ j \neq i}} c_{ij} \, x_{ij}
\]

\textbf{Constraints:}
\begin{align}
\sum_{\substack{j \in V \\ j \neq i}} x_{ij} &= 1, && \forall i \in \{1,\dots,n\} 
\quad &&\text{(each customer is visited exactly once)} \\[4pt]
\sum_{\substack{i \in V \\ i \neq j}} x_{ij} &= 1, && \forall j \in \{1,\dots,n\} 
\quad &&\text{(each customer is left exactly once)} \\[4pt]
\sum_{\substack{j \in V \\ j \neq 0}} x_{0j} &\le K, \quad
\sum_{\substack{i \in V \\ i \neq 0}} x_{i0} \le K 
&& &&\text{(maximum $K$ vehicles)} \\[4pt]
0 \le f_{ij} &\le Q \, x_{ij}, && \forall i,j \in V, \ i \neq j 
\quad &&\text{(linking flow and arc)} \\[4pt]
\sum_{\substack{j \in V \\ j \neq i}} f_{ji} + d_i 
&= \sum_{\substack{j \in V \\ j \neq i}} f_{ij}, && \forall i \in \{1,\dots,n\}
\quad &&\text{(flow conservation at customers)} \\[4pt]
\sum_{\substack{j \in V \\ j \neq 0}} f_{0j} &= \sum_{i=1}^{n} d_i 
\quad &&\text{(total flow leaving the depot)}
\end{align}

Domains:
\[
x_{ij} \in \{0,1\}, \qquad f_{ij} \ge 0
\]

\bigskip

\section{Fitness function and penalized cost for CVRP}
In genetic algorithms, the fitness function evaluates the quality of a candidate solution (set of routes).
Because not all generated solutions satisfy capacity or vehicle-number constraints, a penalized cost is used to integrate these violations into the total cost.

\bigskip

\textbf{Let $C(S)$ be the unpenalized total cost of a solution $S$ (set of routes):}
\[
C(S)\;=\;\sum_{r\in S}\;\sum_{(i,j)\in r} c_{ij},
\]
where the inner sum is over successive arcs in route $r$ (including the arc from depot 0 to the first customer and the arc from the last customer back to 0).

\section*{Route-specific definitions}
For each route $r \in S$, define:
\[
Q_r \;=\; \sum_{i\in r} d_i
\qquad\text{(total load on route r }r\text{)},
\]
\[
\Delta_r \;=\; \max\bigl(0,\;Q_r - Q\bigr)
\qquad\text{(capacity violation for rout }r\text{)}.
\]

If there is a limit $K$ on the number of vehicles, also define:
\[
\Delta_K \;=\; \max\bigl(0,\; |S| - K \bigr)
\qquad\text{(number of vehicles in excess)}.
\]

\section*{Penalized cost}
Then the penalized cost for solution $S$ is:
\[
C_{\mathrm{pen}}(S)
\;=\;
C(S) \;+\; \alpha \sum_{r\in S} \Delta_r \;+\; \beta\,\Delta_K,
\]
where $\alpha,\beta>0$ are penalty coefficients (e.g., $\alpha$ — cost per unit of exceeded load; $\beta$ — fixed cost per vehicle in excess).

\section*{Transformation into a fitness function}
\[
\operatorname{fitness}(S)
\;=\;
\frac{1}{1 + C_{\mathrm{pen}}(S)}.
\]
So, higher fitness values correspond to solutions with lower penalized cost (i.e., shorter and more feasible routes).

\bigskip

\section{Representation of the Individual for CVRP}

In the Genetic Algorithm framework, each individual encodes a possible solution to the Capacitated Vehicle Routing Problem (CVRP). 
Each solution consists of a set of routes that start and end at the central depot, visiting every customer exactly once.

\subsection*{Array-based sequential representation}

In this representation, each gene corresponds to a node: either the depot or a customer. 
The value 0 represents the depot (start or end of a route), while numbers $1,2,3,\dots,n$ represent customers. 

For example:
\[
\begin{aligned}
R_1 &: 0 \rightarrow 2 \rightarrow 5 \rightarrow 7 \rightarrow 0, \\
R_2 &: 0 \rightarrow 3 \rightarrow 4 \rightarrow 8 \rightarrow 0, \\
R_3 &: 0 \rightarrow 1 \rightarrow 6 \rightarrow 9 \rightarrow 0.
\end{aligned}
\]

\subsection*{Advantages}
\begin{itemize}
  \item Clear and intuitive structure: The representation directly shows the order of visits for each vehicle.
  \item Easy to verify route validity and vehicle capacity constraints: You can quickly compute the load of each vehicle.
  \item Single-array representation: All routes are represented in a single array, simplifying storage and manipulation.
  \item Flexible for different VRP variants: Works with CVRP, VRPTW, VRPPD, etc., by adjusting constraints or penalties.
\end{itemize}

\subsection*{Disadvantages}
\begin{itemize}
  \item Sequence interpretation can be less intuitive for very large instances: Long arrays with many vehicles and customers can be harder to read manually.
  \item Fixed depot representation: If multiple depots are involved, the array format needs modification.
  \item Limited direct information about route optimization: The representation shows order, but does not encode distance minimization explicitly.
\end{itemize}

\subsection*{Representation of a CVRP solution using numerical customer IDs}

\begin{figure}[H]
\centering

\caption*{
Each number represents a customer, and 0 represents the depot (start and end of each route). 
The sequence encodes the exact visiting order of customers for each route.
}

\vspace{1cm}

\begin{subfigure}{\textwidth}
\centering
\begin{tikzpicture}[
    every node/.style={circle, draw, thick, minimum size=1cm, font=\bfseries},
    depot/.style={fill=red!80, text=white},
    groupA/.style={fill=cyan!40},
    groupB/.style={fill=orange!70},
    groupC/.style={fill=green!40}
]

\node[depot] (depot) {depot};
\node[groupA] (1) [above left=1.5cm and 1cm of depot] {1};
\node[groupA] (2) [above=1.5cm of depot] {2};
\node[groupA] (7) [above right=1.5cm and 1cm of depot] {7};
\node[groupC] (5) [right=1.5cm of depot] {5};
\node[groupC] (6) [below right=1.5cm and 1cm of depot] {6};
\node[groupB] (4) [below left=1.5cm and 1cm of depot] {4};
\node[groupB] (3) [left=1.5cm of depot] {3};

\draw (depot) -- (1);
\draw (depot) -- (2);
\draw (depot) -- (7);
\draw (depot) -- (3);
\draw (depot) -- (4);
\draw (depot) -- (5);
\draw (depot) -- (6);
\draw (1) -- (2);
\draw (2) -- (7);
\draw (5) -- (6);
\draw (3) -- (4);

\node[below=0.1cm of depot, draw=none, inner sep=0pt] (seq) {
\begin{tikzpicture}[every node/.style={rectangle, draw=black, thick,
                                       minimum width=0.8cm, minimum height=0.8cm,
                                       inner sep=0pt, outer sep=0pt, font=\bfseries},
                    groupA/.style={fill=cyan!40},
                    groupB/.style={fill=orange!70},
                    groupC/.style={fill=green!40}]
]
    \node (a0) {0};
    \node[groupC, right=0pt of a0] (a5) {5};
    \node[groupC, right=0pt of a5] (a6) {6};
    \node[right=0pt of a6] (a00) {0};
    \node[groupA, right=0pt of a00] (a1) {1};
    \node[groupA, right=0pt of a1] (a2) {2};
    \node[groupA, right=0pt of a2] (a7) {7};
    \node[right=0pt of a7] (a01) {0};
    \node[groupB, right=0pt of a01] (a3) {3};
    \node[groupB, right=0pt of a3] (a4) {4};
    \node[right=0pt of a4] (a02) {0};
\end{tikzpicture}
};

\end{tikzpicture}
\vspace{-2.5cm}
\caption*{(1) Visual grouping of customers by color and route}
\end{subfigure}

\label{fig:cvrp_chromosome_clients}
\end{figure}

\newpage

\section{Selection Operators in the Genetic Algorithm}

In a Genetic Algorithm (GA), the selection operator determines which individuals from the current population are chosen to produce offspring for the next generation.
Its role is to favor individuals with higher fitness while still preserving diversity.

\textbf{1. Roulette Wheel Selection}

Roulette Wheel Selection assigns each individual a probability proportional to its fitness.

If the population contains $m$ individuals and the fitness of individual $S_i$ is $f_i$, then:

\[
P(S_i)\;=\frac{f_i}{\sum_{k=1}^m f_k}
\]

\textbf{Advantages:}
\begin{itemize}
\item Simple and widely used.
\item Higher fitness $\Rightarrow$ higher probability of selection.
\end{itemize}

\textbf{Disadvantages:}
\begin{itemize}
\item Premature convergence if a few individuals dominate.
\item Sensitive to large fitness differences.
\end{itemize}

\textbf{2. Rank Selection}

Instead of using raw fitness values, individuals are sorted by fitness.
If the best individual has rank $m$ and the worst has rank $1$, then:

\[
P(S_i)\;=\frac{r_i}{\sum_{k=1}^m k},
\]

where $r_i$ is the rank of individual $i$.

\textbf{Advantages:}
\begin{itemize}
\item Eliminates the negative effect of extreme fitness values.
\item Maintains constant selective pressure.
\end{itemize}

\textbf{Disadvantages:}
\begin{itemize}
\item Requires sorting the population at each generation.
\end{itemize}

\textbf{3. Tournament Selection}

Tournament Selection works by repeatedly choosing a small random subset of individuals and selecting the best from that subset.

The procedure is:
\begin{enumerate}
\item Select $t$ individuals uniformly at random.
\item Choose the one with the highest fitness.
\item Repeat until enough parents are selected.
\end{enumerate}

Typical value: $t = 2, 3 \text{ or } 4$.

\textbf{Advantages:}
\begin{itemize}
\item Very simple and efficient.
\item Selective pressure is easy to control through $t$.
\item Works well even with noisy or irregular fitness distributions.
\end{itemize}

\textbf{Disadvantages:}
\begin{itemize}
\item Large $t$ reduces population diversity.
\end{itemize}

\textbf{4. Elitism}

Elitism means that the best individuals automatically survive to the next generation.

Let:

$E =$ number of elite individuals

Typical values: $E = 1$ or $E = 2$.

\textbf{Advantages:}
\begin{itemize}
\item Guarantees that the best solution is not lost.
\item Increases convergence speed.
\end{itemize}

\textbf{Disadvantages:}
\begin{itemize}
\item Too much elitism can cause premature convergence.
\end{itemize}

\bigskip

\textbf{Selection Method Used in This Project}

In this project, we use:

Tournament selection with size $t = 3$,

combined with:

$E = 1$ elite individual.

Reasons:
\begin{itemize}
\item Tournament selection performs well for routing problems such as VRP.
\item It provides a good balance between exploration and exploitation.
\item Elitism ensures that the best individual found so far is preserved.
\end{itemize}

\section{Mutation Operators for CVRP}

Mutation introduces small random changes in an individual, helping the genetic algorithm avoid premature convergence and explore new regions of the search space.
In CVRP, mutation must preserve the constraint that each customer appears exactly once in the solution.

\textbf{1. Swap Mutation}

Swap Mutation selects two customer positions at random and exchanges them.

Example:
\[
[0,5,7,0,3,4,0]=>[0,5,4,0,3,7,0]
\]

\textbf{Advantages:}
\begin{itemize}
\item Simple and effective.
\item Preserves feasibility automatically (no duplicate customers).
\end{itemize}

\textbf{Disadvantages:}
\begin{itemize}
\item Produces only small local changes.
\end{itemize}

\textbf{2. Insertion Mutation}

This operator removes a customer from one position and reinserts it elsewhere in the sequence.

Example:
\[
[0,5,7,3,4,0]=>[0,5,3,7,4,0]
\]

Procedure:
\begin{enumerate}
\item Choose a random customer.
\item Remove it from its current position.
\item Insert it at another random position (not necessarily in the same route).
\end{enumerate}

\textbf{Advantages:}
\begin{itemize}
\item More flexible than swap.
\item Can create significantly different route structures.
\end{itemize}

\textbf{Disadvantages:}
\begin{itemize}
\item Can disrupt good partial routes.
\end{itemize}

\textbf{3. Inversion Mutation (Reversal)}

Inversion selects a random sub-sequence and reverses its order.

Example:

\[
[0,5,\underline{7,3,4},0]=>[0,5,\underline{4,3,7},0]
\]

\textbf{Advantages:}
\begin{itemize}
\item Very effective for route optimization (similar to 2-opt).
\item Useful for reducing route length.
\end{itemize}

\textbf{Disadvantages:}
\begin{itemize}
\item Mutation effect can be large depending on interval length.
\end{itemize}

\textbf{4. 2-opt Local Search Mutation}

This mutation operator transforms a route into a shorter one by removing two edges and reconnecting the route in a different order.

Given customers $i$ and $j$ in a route, 2-opt replaces:

\[
(i,i+1), (j,j+1)\;with\;(i,j), (i+1,j+1).
\]

Effect: the segment between $i{+}1$ and $j$ is reversed.

\textbf{Advantages:}
\begin{itemize}
\item Strong local optimization operator.
\item Often improves route quality significantly.
\end{itemize}

\textbf{Disadvantages:}
\begin{itemize}
\item More computationally expensive.
\item Typically applied with a small probability.
\end{itemize}

\textbf{5. Route Splitting Mutation (Optional)}

This operator selects two random cut points and reshuffles route boundaries.

Example:

\[
[0,5,7,0,3,4,0]=>[0,5,7,3,0,4,0]
\]

Useful when the algorithm tends to produce too many or too few vehicles.

\textbf{Advantages:}
\begin{itemize}
\item Encourages exploration between routes.
\item Helps escape local optima.
\end{itemize}

\textbf{Disadvantages:}
\begin{itemize}
\item Can easily violate route capacity constraints (fixed later using penalty or repair).
\end{itemize}

\bigskip

\textbf{Mutation Strategy Used in This Project}

For this CVRP implementation, we use a combination of two mutation operators:

\[
Primary: Swap\;Mutation\;(probability\;0.7)
\]

\[
Secondary: Inversion\;Mutation\;(probability\;0.3)
\]

\begin{itemize}
\item Swap is fast and maintains good diversity.
\item Inversion helps improve route structure and reduce travel cost.
\item Both preserve the permutation property (each customer appears exactly once).
\end{itemize}

\section{Survivor Selection Strategies}

In this project, three survivor selection strategies are used. They correspond to classical evolutionary strategies, but are adapted to the CVRP genetic algorithm.

\textbf{1. $(\mu, \lambda)$ Selection (Parents Die, Many Offspring, Best Offspring Survive)}

In this strategy, all parents are removed after reproduction.
Each parent produces several offspring, and then the next generation is created by selecting the best individuals among all offspring.

Let:

\[
\mu=number\;of\;parents,\;\lambda=number\;of\;generated\;offspring.
\]

Only the best $\mu$ offspring survive:

\[
Next\;generation=best\;\mu\;individuals\;from\;the\;offspring.
\]

\textbf{Characteristics:}
\begin{itemize}
\item Parents do not survive.
\item Strong selective pressure on offspring quality.
\item Useful for exploration when producing many children.
\end{itemize}

\textbf{2. $(\mu + \lambda)$ Selection (Parents Survive, Best Individuals from Both Sets)}

Here, parents are allowed to survive together with their offspring.
Selection is performed on the union of parents and offspring:

\[
Next\;generation=best\;\mu\;individuals\;from\;(parents+offspring).
\]

\textbf{Characteristics:}
\begin{itemize}
\item Parents may remain in the population if they are good.
\item Provides stability and prevents losing high-quality solutions.
\item Maintains the same number of parents as the original population.
\end{itemize}

\textbf{3. Fitness-Proportional Offspring Selection (Parents Die, Offspring Chosen Probabilistically)}

In this survival strategy, parents are removed, and offspring are selected to form the next generation using probabilistic selection based on fitness.

Each offspring $S_i$ has a probability of survival proportional to its fitness $f_i$:

\[
P(S_i)=\frac{f_i}{\sum_{k=1}^\lambda f_k}
\]

The algorithm repeatedly samples individuals according to these probabilities until $\mu$ survivors are chosen.

\textbf{Characteristics:}
\begin{itemize}
\item Parents die; only offspring can enter the next generation.
\item Selection is randomized but biased toward fitter individuals.
\item Helps maintain diversity compared to purely deterministic selection.
\end{itemize}

\section{Bibliografie}
\url{https://blog.locus.sh/vehicle-routing-problem/}

\url{https://en.wikipedia.org/wiki/Vehicle_routing_problem}

\url{https://tttp-au.com/wp-content/uploads/2024/06/Pages-from-TTTP-Vol-9_No-1_2024-WEB-8.pdf}

\url{https://ro.wikipedia.org/wiki/Problema_rut%C4%83rii_vehiculelor}

\url{https://ro.wikipedia.org/wiki/Problema_comis-voiajorului}

\url{https://developers.google.com/optimization/routing/cvrp}

\url{https://www.upperinc.com/glossary/route-optimization/vehicle-routing-problem-with-profits-vrpp/}

\url{https://developers.google.com/optimization/routing/vrptw}

\url{https://www.sciencedirect.com/science/article/abs/pii/S0305054825000085}

\url{https://www.sciencedirect.com/science/article/abs/pii/S0305054820301040}

\url{https://www.upperinc.com/glossary/route-optimization/vehicle-routing-problem-with-backhauls-vrpb/}

\url{https://www.upperinc.com/glossary/route-optimization/periodic-vehicle-routing-problem-pvrp/}

\url{https://www.upperinc.com/glossary/route-optimization/stochastic-vehicle-routing-problem-svrp/}

\url{https://www.sciencedirect.com/science/article/pii/S0305054814002159}

\url{https://www.sciencedirect.com/science/article/abs/pii/S0305054813001408}

\url{https://www.hexaly.com/templates/inventory-routing-problem-irp}

\url{https://www.upperinc.com/glossary/route-optimization/open-vehicle-routing-problem-ovrp/}

\url{https://www.sciencedirect.com/topics/mathematics/orienteering-problem}

\url{https://drops.dagstuhl.de/storage/01oasics/oasics-vol014-atmos2010/OASIcs.ATMOS.2010.142/OASIcs.ATMOS.2010.142.pdf}

\url{https://www.researchgate.net/publication/256022425_The_Capacitated_Team_Orienteering_Problem_A_Bi-Level_Filter-and-Fan_Method}

\url{https://www.praiseworthyprize.org/jsm/index.php?journal=irecos&page=article&op=view&path%5B%5D=18244}

\url{https://www.sciencedirect.com/science/article/pii/S2352146516000089}

\url{https://www.researchgate.net/figure/A-CVRP-solution-and-the-individual-encoding-representation-Each-color-represents-a_fig2_394965948}

\end{document}